\documentclass[11pt,twoside]{article}
\usepackage[toc,page,header]{appendix}
\usepackage{pdfpages}
\usepackage{csquotes}
\usepackage{epigraph}
\usepackage{changepage}
\usepackage{fontspec}
\defaultfontfeatures{Scale=MatchLowercase}
\setmainfont[Mapping=tex-text]{Times New Roman}
\setsansfont[Mapping=tex-text]{Arial}
\setmonofont{Courier}

\usepackage{float}
\usepackage{turnstile}
\usepackage{bussproofs}

\usepackage{geometry}
\geometry{letterpaper}

\newtheorem{theorem}{Theorem}
%\newtheorem{cor}{Corollary}
%\newtheorem{lem}{Lemma}
%\theoremstyle{remark}
\newtheorem{remark}{Remark}

\newtheorem{objection}{Objection}
\newenvironment*{response}[1][]{\noindent
\textbf{Response to Objection #1.}
\begin{adjustwidth}{1em}{1em}
}
{\end{adjustwidth}
\vspace{1ex}
}


%\usepackage[parfill]{parskip}    % Activate to begin paragraphs with an empty line rather than an indent

\usepackage{graphicx}
\usepackage[leftcaption]{sidecap}
\sidecaptionvpos{figure}{c}

%\usepackage{amssymb}

\usepackage{epstopdf}
\DeclareGraphicsRule{.tif}{png}{.png}{`convert #1 `dirname #1`/`basename #1 .tif`.png}

\usepackage[
bibstyle=reading,%%numeric,
entrykey=false,
citestyle=authortitle,
natbib=true,
hyperref,
bibencoding=utf8,backref=true,backend=biber]{biblatex}

\usepackage{hyperref}
\hypersetup{
    bookmarks=true,         % show bookmarks bar?
    unicode=true,          % non-Latin characters in Acrobat’s bookmarks
    pdftoolbar=true,        % show Acrobat’s toolbar?
    pdfmenubar=true,        % show Acrobat’s menu?
    pdffitwindow=false,     % window fit to page when opened
    pdfstartview={FitH},    % fits the width of the page to the window
    pdftitle={Discursive Practices},    % title
    pdfauthor={G. A. Reynolds},     % author
    pdfsubject={Subject},   % subject of the document
    pdfcreator={G. A. Reynolds},   % creator of the document
    pdfproducer={Producer}, % producer of the document
    pdfkeywords={keyword1} {key2} {key3}, % list of keywords
    pdfnewwindow=true,      % links in new window
    colorlinks=true,       % false: boxed links; true: colored links
    linkcolor=blue,          % color of internal links
    citecolor=blue,        % color of links to bibliography
    filecolor=magenta,      % color of file links
    urlcolor=cyan           % color of external links
}
\usepackage{draftwatermark}


\usepackage{fancyhdr}
\setlength{\headheight}{15.2pt}
\pagestyle{fancy}

\lhead[Discursive Practices]{\thepage}
\chead[]{}
\rhead[\thepage]{Discursive Practices}

\title{Speech, Discourse, Language \\
\vspace{12pt} \Large{A survey of contemporary models and their relevance to \\
\SR{}}}
\author{G. A. Reynolds}
\date{\today}
\bibliography{%
../bib/abstracts.bib,%
../bib/anthro.bib,%
../bib/biology.bib,%
../bib/brandom.bib,%
../bib/causality.bib,%
../bib/cognitivism.bib,%
../bib/dialogics.bib,%
../bib/embodiment.bib,%
../bib/em.bib,%
../bib/linguistics.bib,%
../bib/logic.bib,%
../bib/mind.bib,%
../bib/philosophy.bib,%
../bib/pragmatism.bib,%
../bib/psychomet.bib%
../bib/psychometrics.bib,%
../bib/misc.bib,%
../bib/measurement.bib,%
../bib/psychology.bib,%
../bib/science.bib,%
../bib/surveyresearch.bib,%
../bib/variables.bib,%
../bib/val.bib,%
../bib/validity.bib,%
}

%% Macros

\newcommand{\SM}{Standard Model}
\newcommand{\XSM}{Extended Standard Model}

\newcommand{\SMeth}{Survey Methodology}

\newcommand{\SR}{Survey Research}
\newcommand{\sr}{survey research}
\newcommand{\SRIV}{Survey Interview}
\newcommand{\sriv}{survey interview}
\newcommand{\SIV}{Survey Interviewing}
\newcommand{\FI}{Field Interviewer}
\newcommand{\Iver}{Interviewer}
\newcommand{\R}{Respondent}
\newcommand{\LPR}{Legal Permanent Resident}
\newcommand{\ART}{Assimilated Response Technique}
\newcommand{\GAM}{Grouped Answer Method}
\newcommand{\IOM}{Instrument of Measurement}

\newcommand{\MIE}{\textit{MIE}}

\includeonly{%
%% pilots,cards
}

\setlength{\epigraphwidth}{4in}

%%%%%%%%%%%%%%%%%%%%%%%%%%%%%%%%%%%%%%%%%%%%%%%%%%%%%%%%%%%%%%%%
\begin{document}
\maketitle
\nocite{*}

\begin{abstract}
abstract
\end{abstract}

\tableofcontents
\listoffigures

\newpage
%%%%%%%%%%%%%%%%%%%%
\section{Introduction}

\begin{abstract}

\end{abstract}

\begin{remark}
  Language is culture, not nature.  It has a history, not an essence.
\end{remark}

Personal v. subpersonal explanations.  Most cognitive sciences focus
on the subpersonal.  For reasons that will become clear, we do not
address this perspective in detail in this paper, but here we provide
an overview of the main issues and prominent approaches.

The main reason for skipping cognitivism is that it (usually) fails to
acknowledge the distinction between the space of reasons and the
natural space of laws.  It may have a lot to say about the underlying
processes involved in cognition, but it cannot \textit{explain} the
distinctively social and rational character of cognition.  It can't
handle \textit{meaning}.  For a detailed analysis of this problem see
``Critique of Cognitive Interviewing''.

\subsection{Speech and Language}

\subsection{Linguistics}

\cite{love_cognition_2004} - an integrationist attack on
language-as-code and ``classical'' view of language.

See also Bakhtin on the utterance.

\subsection{Cognitivist Models}

\cite{levinson_cognition_2006}

\subsection{Neuroscience}

A hot topic today is the relation of neurology to cognition.
Reductionists attempt to show how cognition can be reduced to the
physical entities, states, and processes of the brain.  Etc.

%%%%%%%%%%%%%%%%%%%%
\section{Ethnomethodology and Conversation Analysis}

\begin{abstract}

\end{abstract}

%%%%%%%%%%%%%%%%%%%%
\section{Dialogical Models: Bakhtin and the Scandanavian School}

\begin{remark}
  There are several ``schools'' in this region: Dialogicism (Bahktin,
  Linell, et al.); Integrationism (Roy Harris); Contextualism (who?);
  Interactionalism (who?)  ``Integrationism'' is more closely aligned
  with linguistics, dialogism with discourse studies (and psych,
  anthro, etc.)  I'm not sure if Contextualism and Interactionism have
  been put forth as coherent doctrines by any major figure, but lots
  of people in \SR{} appeal to one or the other or both in some form.
\end{remark}

\begin{abstract}

\end{abstract}

\begin{remark}
  Bakhtin: ``prosaics''
\end{remark}

Bakhtin: constitutive features of the utterance as a unit of speech
communication, features distinguishing it from the units of language:

\begin{itemize}
\item change of speaking subjects (CA: turns) (p. 71, 76)
\item finalization of the utterance (p. 76); criteria:
\begin{itemize}
\item possibility of responding to it (assumption of responsive attitude toward it)
\end{itemize}
finalized wholeness of the utterance determined by 3 aspects/factors (p. 76)
\begin{itemize}
\item referential and semantic exhaustiveness of the theme of the utterance (p. 77)
\item speaker's plan or speech will (p. 77)
\item typical composition and generic forms of finalization (p. 78) - most important determining factor
\end{itemize}
\item ``third feature of the utterance -- the relation of the
  utterance to the \textit{speaker himself} (the author of the
  utterance) adn to the \textit{other} participants in speech
  communication'' (p.84)
\item ``first aspect of the utterance that determines its
  compositional and stylistic features'' is ``referentially semantic
  assignments (plan) of the speech subject (or author) (p.84)
\item ``The second aspect of the utterance that determines its
  composition and style is the \textit{expressive} aspect, taht is,
  the speaker's subjective emotional evaluation of the referentially
  semantic content of his utterance''. p. 84
\end{itemize}

\begin{remark}
  ``So the expressive aspect is a constitutive feature of the utterance.''

``Thus, the utterance, its styule, and its composition are determined
  by its referentially semantic element (the theme) and its expressive
  aspecct, that is, the speaker's evaluative attitude toward the
  referntially semantic element of the utterance.  Stylistics knows no
  third aspect... Such is the prevailing idea'' p. 90

But to this Bakhtin adds: responsivity and addressivity.  ``Any
concrete utterance is a link in the chain of speech communication of a
particular sphere'' etc. p. 91

Responsivity: ``the expression of an utterance can never be fully
understood or explained if its thematic content is all that is taken
into account.  The expression of an utterance always \textit{responds}
to a greater or lesser degree, that is, it expresses the speaker's
attitude toward others' utterance and not just his attitude toward the
object of his utterance... The utterance is filled with
\textit{dialogic overtones}...'' p. 92

Addressivity: ``But the utterance is related not only to preceding,
but also to subsequent links in the chain of speech communion...
[F]rom the very beginning, the utterance is constructed while taking
into account possible responsive reactions, for whose sake, in
essence, it is actually created.'' p. 94

``An essential (constitutive) marker of the utterance is its quality
of being directed to someone, its \textit{addressivity}... Both the
composition and, particularly, the style of the uttearance depend on
those to whom the utterance is addressed, ow the speaker (or writer)
senses and imagines his addressees, and the force of their effect on
the utterance.  Each speech genre in each area of speech communication
has its own typical conception of the addressee, and this defines it
as a genre.''  p. 95
\end{remark}


\cite{bakhtin_problem_1986}

\cite{junefelt_proceedings_2009}

\cite{linell_respect_2009}

\cite{linell_rethinking_2009}

\cite{markova_coding_2007}

\cite{linell_communicative_2010}

\cite{linell_asymmetries_1991}

%%%%%%%%%%%%%%%%%%%%
\section{Deontic Scorekeeping}

\begin{abstract}

\end{abstract}

%%%%%%%%%%%%%%%%

\clearpage
\appendix
\begin{appendices}
\section{Bibliography}
%% \addcontentsline{toc}{chapter}{Bibliography}
%% \bibliographystyle{plainnat}
\printbibliography[heading=none]
\end{appendices}

\end{document}
