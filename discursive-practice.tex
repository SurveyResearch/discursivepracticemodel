\documentclass[11pt,twoside]{article}
\usepackage[toc,page,header]{appendix}
\usepackage{pdfpages}
\usepackage{csquotes}
\usepackage{epigraph}
\usepackage{changepage}
\usepackage{fontspec}
\defaultfontfeatures{Scale=MatchLowercase}
\setmainfont[Mapping=tex-text]{Times New Roman}
\setsansfont[Mapping=tex-text]{Arial}
\setmonofont{Courier}

\usepackage{float}
\usepackage{turnstile}
\usepackage{bussproofs}

\usepackage{geometry}
\geometry{letterpaper}

\newtheorem{theorem}{Theorem}
%\newtheorem{cor}{Corollary}
%\newtheorem{lem}{Lemma}
%\theoremstyle{remark}
\newtheorem{remark}{Remark}

\newtheorem{objection}{Objection}
\newenvironment*{response}[1][]{\noindent
\textbf{Response to Objection #1.}
\begin{adjustwidth}{1em}{1em}
}
{\end{adjustwidth}
\vspace{1ex}
}


%\usepackage[parfill]{parskip}    % Activate to begin paragraphs with an empty line rather than an indent

\usepackage{graphicx}
\usepackage[leftcaption]{sidecap}
\sidecaptionvpos{figure}{c}

%\usepackage{amssymb}

\usepackage{epstopdf}
\DeclareGraphicsRule{.tif}{png}{.png}{`convert #1 `dirname #1`/`basename #1 .tif`.png}

\usepackage[
bibstyle=reading,%%numeric,
entrykey=false,
citestyle=authortitle,
natbib=true,
hyperref,
bibencoding=utf8,backref=true,backend=biber]{biblatex}

\usepackage{hyperref}
\hypersetup{
    bookmarks=true,         % show bookmarks bar?
    unicode=true,          % non-Latin characters in Acrobat’s bookmarks
    pdftoolbar=true,        % show Acrobat’s toolbar?
    pdfmenubar=true,        % show Acrobat’s menu?
    pdffitwindow=false,     % window fit to page when opened
    pdfstartview={FitH},    % fits the width of the page to the window
    pdftitle={Discursive Practices},    % title
    pdfauthor={G. A. Reynolds},     % author
    pdfsubject={Subject},   % subject of the document
    pdfcreator={G. A. Reynolds},   % creator of the document
    pdfproducer={Producer}, % producer of the document
    pdfkeywords={keyword1} {key2} {key3}, % list of keywords
    pdfnewwindow=true,      % links in new window
    colorlinks=true,       % false: boxed links; true: colored links
    linkcolor=blue,          % color of internal links
    citecolor=blue,        % color of links to bibliography
    filecolor=magenta,      % color of file links
    urlcolor=cyan           % color of external links
}
\usepackage{draftwatermark}


\usepackage{fancyhdr}
\setlength{\headheight}{15.2pt}
\pagestyle{fancy}

\lhead[Discursive Practices]{\thepage}
\chead[]{}
\rhead[\thepage]{Discursive Practices}

\title{Speech, Discourse, Language \\
\vspace{12pt} \Large{A survey of contemporary models and their relevance to \\
\SR{}}}
\author{G. A. Reynolds}
\date{\today}
\bibliography{%
../bib/abstracts.bib,%
../bib/anthro.bib,%
../bib/biology.bib,%
../bib/brandom.bib,%
../bib/causality.bib,%
../bib/cognitivism.bib,%
../bib/coginterviewing.bib,%
../bib/dialogics.bib,%
../bib/embodiment.bib,%
../bib/em.bib,%
../bib/linguistics.bib,%
../bib/logic.bib,%
../bib/mind.bib,%
../bib/philosophy.bib,%
../bib/pragmatism.bib,%
../bib/psychomet.bib%
../bib/psychometrics.bib,%
../bib/misc.bib,%
../bib/measurement.bib,%
../bib/psychology.bib,%
../bib/science.bib,%
../bib/surveyresearch.bib,%
../bib/variables.bib,%
../bib/val.bib,%
../bib/validity.bib,%
}

%% Macros

\newcommand{\SM}{Standard Model}
\newcommand{\XSM}{Extended Standard Model}

\newcommand{\SMeth}{Survey Methodology}

\newcommand{\SR}{Survey Research}
\newcommand{\sr}{survey research}
\newcommand{\SRIV}{Survey Interview}
\newcommand{\sriv}{survey interview}
\newcommand{\SIV}{Survey Interviewing}
\newcommand{\FI}{Field Interviewer}
\newcommand{\Iver}{Interviewer}
\newcommand{\R}{Respondent}
\newcommand{\LPR}{Legal Permanent Resident}
\newcommand{\ART}{Assimilated Response Technique}
\newcommand{\GAM}{Grouped Answer Method}
\newcommand{\IOM}{Instrument of Measurement}

\newcommand{\MIE}{\textit{MIE}}

\includeonly{%
%% pilots,cards
}

\setlength{\epigraphwidth}{4in}

%%%%%%%%%%%%%%%%%%%%%%%%%%%%%%%%%%%%%%%%%%%%%%%%%%%%%%%%%%%%%%%%
\begin{document}
\maketitle
\nocite{*}

\begin{abstract}
A common lament in the \SR{} literature is the lack of a good model of
what is variously referred to as the ``survey interview process'', the
``question-answer process'', the ``response process'', or the like.
Where researchers do articulate an explicit model, they tend to rely
on the sort of cognitivist model exemplified by \cite{tourangeau_psr}.

But today we have a great variety of distinct models of speech,
discourse, and language.  The purpose of this paper is to critically
survey some of the best known such models and examine their relevance
to survey research interviewing.

The first part of this paper provides background.  It begins with a
brief overview of the critical distinction between the natural ``space
of laws'' and the cultural ``space of reasons''.  Discourse is
obviously dependent on the causal realm; it's hard to talk without a
body, or to imagine a mind without a brain.  Yet the
\textit{intelligibility} of discursive behaviour (speech, language)
seems to call for a distinctive order of explanation, one that swings
free of cause and effect and instead appeals to notions of normativity
and rationality.  This first section examines the tension between
these orders of explanation and provides a general overview of some of
the distinct (and sometimes incompatible) ways of addressing it.  It
concludes that any model adequate to the needs of survey research
should disregard the causal realm of the states and processes
underlying discursive practices, and instead focus on the rational
structure of those practices - what Wilfrid Sellars dubbed ``the space
of reasons''.

This section also provides a brief overview of the dominant modes of
linguistic thought in the 20th century, with particular attention to
Chomskyism.  The main purpose of the overview is therapeutic:
Chomskyism is basically dead, but, zombie-like, it refuses to die.
Many of its tenets (competence v. performance, language acquisition
v. language learning, Universal Grammar) are still defended by some
specialists (e.g. Pinker) and continue to enjoy uncritical acceptance
by non-specialists (including in particular survey research
methodologists).  So one purpose of this section is to expose the
problems with 20th century ``scientific'' linguistics in general and
Chomskyism in particular.

Finally, it sketches some of the main relevant themes from cognitive
science, neuroscience, and the philosophy of language.


The second part examines three (four?) distinctive approaches to the
study of discursive practice.

\textit{Ethnomethodology} and its offshoot \textit{Conversation
  Analysis} seek to understand such behavior in terms of the local,
accountable order actively produced by participants in discourse.  It
reverses the standard sociological order of explanation, which seeks
to understand the doings of individuals in terms of causal forces
exerted by social entities and processes.  CA instead examines the
fine detail of actual situated episodes of discursive behavior in
order to discover how participants (``members'') manage to produce and
sustain as a local, situated phenomenon.

\textit{Dialogism} the rubric adopted by a number of scholars (mainly
northern Europeans in psychology departments) for a framework or
collection of doctrines traceable (mainly) to Mikhail Bakhtin.  It is
largely motivated by Bakhtin's observation that \textit{utterance} is
essentially dialogical; it always presupposes not only a speaker, but
also \textit{responsivity} and \textit{addressivity}.  This approach
categorically rejects the atomistic, monological perspective that
usually characterizes cognitivist approaches.  To understand
discursive episodes, one must understand a complex whole in which the
parts (individual utterances) are always essentially interrelated.

\textit{Integrationism} is an approach to linguistics advocated by the
linguist Roy Harris in reaction to the sort of structuralist,
cognitivist conceptions of linguistics that have dominated the field
since the days of de Saussure.  In particular, it rejects what Harris
calls the ``telementation model'' of language, according to which
discourse is reducible to the encoding, transmissioin, and decoding of
thought.  Although this is primarily a model of linguistics, it is
closely related to dialogical models and has direct relevance to the
study of discursive practices.

\textit{Pragmatism} begins by asking what counts as discursive
\textit{practice}.  Instead of asking ``what is it?'', it asks
questions like ``what role does it play in our lives?'' and ``what
must one \textit{do} in order to count as deploying it?'', etc. (Huw
Price, et al.)  The most thoroughly worked out pragmatist model of
discursive practice is the \textit{deontic scorekeeping} model.  This
is a very sophisticated account of the pragmatic foundations of
discursive (and thus conceptual) practice, elaborated by the
philosopher Robert Brandom in his 1994 masterpiece
\citetitle{brandom_mie}.

One striking fact emerges: these approaches may employ distinctive
vocabularies, yet they are all clearly talking about more-or-less the
same sorts of things.  They are recognizable as variants on a few
master concepts: the primacy of practice (and hence of empirical
investigation over speculative theorizing); the situated or
context-dependence of meaning; the essentially social nature of
language, thought, and communication; etc.  (Another way to put this
might be to say that they share a common enemy, one that appears in a
variety of guises: monologism, cartesianism, atomism,
representationism, etc.)

\end{abstract}

\tableofcontents
\listoffigures

\newpage
%%%%%%%%%%%%%%%%%%%%
\section{Introduction}

\begin{abstract}

\end{abstract}

\begin{remark}
  Language is culture, not nature.  It has a history, not an essence.
\end{remark}

Personal v. subpersonal explanations.  Most cognitive sciences focus
on the subpersonal.  For reasons that will become clear, we do not
address this perspective in detail in this paper, but here we provide
an overview of the main issues and prominent approaches.

The main reason for skipping cognitivism is that it (usually) fails to
acknowledge the distinction between the space of reasons and the
natural space of laws.  It may have a lot to say about the underlying
processes involved in cognition, but it cannot \textit{explain} the
distinctively social and rational character of cognition.  It can't
handle \textit{meaning}.  For a detailed analysis of this problem see
``Critique of Cognitive Interviewing''.

\subsection{Linguistics}


\begin{remark}
  What about linguistics?  Should we list it as a distinctive model of
  discursive practice?  The problem is messiness; the Chomskyite
  school is arguably a branch of cognitive science, but there are very
  many distinct kinds of linguistics.  In the past few decades,
  linguistics has seen vigorous attempts to break free from the
  straightjacket of Chomskyite thinking.  (Some examples:
  \cite{evans_myth_2009}; \cite{christiansen_myth_2009};
  \cite{tomasello_language_1995}; \cite{tomasello_universal_2009};
  \cite{minsky_chomsky_irrelevance}.  Should we include here an
  overview of the (pernicious?) influence of Chomsky on linguistic
  thought, and a survey of the correctives that have begun to heal the
  patient (e.g. emergent grammar, actual empirical studies of speech,
  etc.)?  That would arguably be beyond the scope of this paper, but
  on the other hand, any account of discursive practice must have
  something to say about the deliverances of modern ``linguistic
  science'' - even if only to say that it's all BS
  (\cite{baker_language_1986}).  At the least, we need to say
  something about linguistics, since many readers will have at least
  some acquaintance with some of the theories of linguistics, so we
  need to provide some guidance as to how to position those ideas with
  respect to the other models discussed here.

  Another good reason to include a substantial section on linguistics
  is that the models of real interest to us -- those that take
  seriously issues like context, interactivity, practice etc. -- stand
  it conspicuous contrast to the linguistic theories that have
  dominated the discipline since at least the late 1950s.  So there is
  a cautionary tale to be told: classic modern linguistics went off

  the rails in various ways, which we need to make explicit (e.g. it
  distrusted or even dismissed empirical research, etc.)

  Another point: Chomskyism is basically dead for serious researchers,
  but, zombie-like, it refuses to die.  Some specialists (Pinker,
  etc.) continue to defend the basic tenets of Chomskyism, and
  non-specialists routinely and uncritically accept such myths as the
  competence-performance distinction, deep structure, and Universal
  Grammar.  Because these doctrines (dogmas, really) are so deeply and
  widely entrenched in the popular mind, they deserve special
  attention.
\end{remark}


\subsubsection{Historical and Comparative}

\subsubsection{``Scientific'': from de Saussure to Chomsky}

\cite{love_cognition_2004} - an integrationist attack on
language-as-code and ``classical'' view of language.

See also Bakhtin on the utterance.

\subsection{Cognitivism}

\cite{levinson_cognition_2006}

\subsection{Neuroscience}

A hot topic today is the relation of neurology to cognition.
Reductionists attempt to show how cognition can be reduced to the
physical entities, states, and processes of the brain.  Etc.

\subsection{Philosophy of Language}

\subsection{Speech and Language}

%%%%%%%%%%%%%%%%%%%%
\section{Ethnomethodology and Conversation Analysis}

\begin{abstract}

\end{abstract}

%%%%%%%%%%%%%%%%%%%%
\section{Dialogisim}

\begin{remark}
  There are several ``schools'' in this region: Dialogicism (Bahktin,
  Linell, et al.); Integrationism (Roy Harris); Contextualism (who?);
  Interactionalism (who?)  ``Integrationism'' is more closely aligned
  with linguistics, dialogism with discourse studies (and psych,
  anthro, etc.)  I'm not sure if Contextualism and Interactionism have
  been put forth as coherent doctrines by any major figure, but lots
  of people in \SR{} appeal to one or the other or both in some form.
\end{remark}

\begin{abstract}

\end{abstract}

\subsection{Bakhtin}
\begin{remark}
  Bakhtin: ``prosaics''
\end{remark}

Bakhtin: constitutive features of the utterance as a unit of speech
communication, features distinguishing it from the units of language:

\begin{itemize}
\item change of speaking subjects (CA: turns) (p. 71, 76)
\item finalization of the utterance (p. 76); criteria:
\begin{itemize}
\item possibility of responding to it (assumption of responsive attitude toward it)
\end{itemize}
finalized wholeness of the utterance determined by 3 aspects/factors (p. 76)
\begin{itemize}
\item referential and semantic exhaustiveness of the theme of the utterance (p. 77)
\item speaker's plan or speech will (p. 77)
\item typical composition and generic forms of finalization (p. 78) - most important determining factor
\end{itemize}
\item ``third feature of the utterance -- the relation of the
  utterance to the \textit{speaker himself} (the author of the
  utterance) adn to the \textit{other} participants in speech
  communication'' (p.84)
\item ``first aspect of the utterance that determines its
  compositional and stylistic features'' is ``referentially semantic
  assignments (plan) of the speech subject (or author) (p.84)
\item ``The second aspect of the utterance that determines its
  composition and style is the \textit{expressive} aspect, taht is,
  the speaker's subjective emotional evaluation of the referentially
  semantic content of his utterance''. p. 84
\end{itemize}

\begin{remark}
  ``So the expressive aspect is a constitutive feature of the utterance.''

``Thus, the utterance, its styule, and its composition are determined
  by its referentially semantic element (the theme) and its expressive
  aspecct, that is, the speaker's evaluative attitude toward the
  referntially semantic element of the utterance.  Stylistics knows no
  third aspect... Such is the prevailing idea'' p. 90

But to this Bakhtin adds: responsivity and addressivity.  ``Any
concrete utterance is a link in the chain of speech communication of a
particular sphere'' etc. p. 91

Responsivity: ``the expression of an utterance can never be fully
understood or explained if its thematic content is all that is taken
into account.  The expression of an utterance always \textit{responds}
to a greater or lesser degree, that is, it expresses the speaker's
attitude toward others' utterance and not just his attitude toward the
object of his utterance... The utterance is filled with
\textit{dialogic overtones}...'' p. 92

Addressivity: ``But the utterance is related not only to preceding,
but also to subsequent links in the chain of speech communion...
[F]rom the very beginning, the utterance is constructed while taking
into account possible responsive reactions, for whose sake, in
essence, it is actually created.'' p. 94

``An essential (constitutive) marker of the utterance is its quality
of being directed to someone, its \textit{addressivity}... Both the
composition and, particularly, the style of the uttearance depend on
those to whom the utterance is addressed, ow the speaker (or writer)
senses and imagines his addressees, and the force of their effect on
the utterance.  Each speech genre in each area of speech communication
has its own typical conception of the addressee, and this defines it
as a genre.''  p. 95
\end{remark}

\cite{bakhtin_problem_1986}

\cite{junefelt_proceedings_2009}

\subsection{The Scandanavians}

Bakhtin's notions of dialogicality seems to have had a major influence
among a group of European psychologists, esp. Ragnar Rommetveit
(Norway), Per Linell (Sweden), Ivana Marková (Czech, working in
Scotland).  Linell (at least; I haven't read much Marková) has
attempted to articulate a more or less explicit articulation of
dialogism as a framework (or at least coherent collection of related
doctrines) for the study of language, communication, the self, etc.
These scholars are psychologists, but their kind of psychology is
markedly different from the American kind.  (For one thing, they waste
little or no time trying to mimic the physical sciences.)  But this
should not be taken as an indication the dialogism is a pychological
doctrine or theory; it is much broader than that.  These scholars
\textit{use} dialogism as a framework within which to investigate
psychology (and other).

Dialogism -- as presented by Linell, at any rate -- has clear
affinities with contemporary pragmatism of the sort practiced by
Rorty, Brandom, Price, etc.

\cite{linell_respect_2009}

\cite{linell_rethinking_2009}

\cite{markova_coding_2007}

\cite{linell_communicative_2010}

\cite{linell_asymmetries_1991}

%%%%%%%%%%%%%%%%%%%%
\section{Pragmatism}

\subsection{Overview: pragmatist perspectives on discourse}

\subsection{Deontic Scorekeeping}

\begin{abstract}

\end{abstract}


\begin{remark}
Why the deontic scorekeeping model is preferable to others, esp. the
cognitive model.
\end{remark}

\begin{remark}
  It's a model of discursive, that is rational, practice.  Contrast
  this with most models on offer which tend to focus on subpersonal
  processes; hence the prevalence of talk about ``the survey
  process'', the ``response process'', etc.
\end{remark}


%%%%%%%%%%%%%%%%

\clearpage
\appendix
\begin{appendices}
\section{Bibliography}
%% \addcontentsline{toc}{chapter}{Bibliography}
%% \bibliographystyle{plainnat}
\printbibliography[heading=none]
\end{appendices}

\end{document}
